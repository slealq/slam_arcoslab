% ----------------------
  \chapter{Introducción}
% ----------------------
\label{C:introduccion}



%--------------------------------------------------------------------
Desde los años sesenta, con la revolución industrial, los humanos han sido relevados de operar tareas riesgosas. La nueva tecnología desarrollado ha evolucionado, siendo utilizada posteriormente para realizar tareas más complejas, o de mucha precisión \cite{Garcia2007}. La fabricación de micro procesadores, o autos en masa son algunos ejemplos de productos que se han florecido gracias a esta tecnología.

Sin embargo, actualmente existen nuevas necesidades y mercados fuera de la tradicional manufactura. Por ejemplo, limpieza, cocina, construcción, agricultura, entre otras. Esto demanda una nueva generación de robots que puedan satisfacer estas necesidades \cite{Garcia2007}.

La robótica es la ciencia de percibir y manipular el mundo físico através de dispositivos mecánicos controlados por computadoras \cite{Thrun2005}. Hasta el día de hoy la robótica ha logrado perfeccionar los movimientos precisos milimétricamente, y las tareas repetitivas. Sin embargo, fuera de una línea de producción, estas ventajas no son útiles, pues por ejemplo en un hogar los ambientes son dinámicos y el robot por lo tanto debe poseer características cognitivas \cite{HelioAzevedoRenatoArcherITCenterandICM/USPJosePedroR.InstituteofMathematicalandComputerSciences2017}. 

El laboratorio de investigación ARCOS-Lab (Autonomous robots and cognitive systems) se enfoca primariamente en desarrollar e implementar nuevas técnicas en dispositivos mecánicos con el fin de satisfacer algunas de las necesidades mencionadas anteriormente.

Actualmente (2018) el ARCOS-Lab está trabajando en la construcción de un robot humanoide, el cuál se desea que posea varias de las características mencionadas anteriormente (reconocer su ambiente, y realizar tareas dinámicas como cocinar).

La navegación es uno de los aspectos más importantes que se debe desarrollar para cumplir el fin propuesto. Por lo tanto, el presente proyecto plantea implementar un algoritmo llamado SLAM (Simultaneous localization and mapping) en la navegación dos-dimensional de un robot omnidireccional con ruedas mecanum. 

Una de las herramientas más potentes de hoy en día en el tema de la robótica es un sistema llamda ROS (Robot Operating System). El mismo se desarrolla primariamente en Linux, y posee muchas herramientas necesarias para realizar una tarea como la propuesta. El software ayuda a incorporar modularmente sensores y algoritmos de software desarrollados por científicos que trabajan en el área.


\section{Objetivos}

\subsection{Objetivo general}
Dotar a un robot omnidireccional con ruedas mecanum la capacidad de mapeo y localización simultánea.

\subsection{Objetivos específicos}
Para el desarrollo de este proyecto se establecieron los siguientes objetivos:

\begin{itemize} % lista con viñetas
	\item Desarrollar el software de control para los motores de las ruedas y los sensores de odometría.
	\item Desarrollar la infraestructura de comunicación entre el Raspberry Pi y el cuerpo del robot (motores y odometría). \item Implementar el software de comunicación y visualización del sensor de profundidad.
	\item Integrar todos los módulos del sistema (software y hardware) y realizar pruebas de conectividad y funcionamiento para cada módulo del sistema.
	\item Implementar el sistema de SLAM en la computadora principal.
	\item Determinar y ejecutar experimentos que permitan validar el funcionamiento completo de todo el sistema para realizar pruebas de mapeo y localización simultánea.
\end{itemize}

\section{Metodología}

\begin{enumerate}  %lista numerada
    \item Conectar el Kinect con una computadora corriendo Linux e utilizar ROS (Robot Operating System) para ver la imagen producida por la cámara. 
    \item Convertir la imagen en formato PointCloud2 (3d) a un formato LaserScan (2d).
    \item Conectar el sensor láser Hokuyo a una computadora corriendo Linux e utilizar ROS para interpretar la imagen en formato LaserScan.
    \item Utilizar el simulador Gazebo en ROS con un modelo básico de un robot, e probar el uso de sensores y movimiento del mismo.
    \item Conectar el controlador de motores RoboClaw con un STM32F4 y hacer pruebas de comunicación. 
    \item Realizar pruebas de movimiento, e implemetar el protocolo de comunicación en el STM32f4.
    \item Utilizar un Joystick DualShock 4 en ROS y utilizarlo para mover el robot.
    \item Implementar el algoritmo de gmapping a el robot de la simulación en gazebo. 
    \item Implementar el algoritmo de localización y navegación amcl en el simulador gazebo. 
    \item Utilizar una Raspberry Pi para mandarle información de movimiento al STM32F4-Roboclaw. Mover los motores y hacer pruebas. 
    \item Realizar el cálculo de odometría con el Raspberry Pi con la información dada por el RoboClaw. 
    \item Confeccionar el módulo que conecta el STM32F4 con ROS.
    \item Trasladar lo implementado en Gazebo hacia el robot real.
    \item Realizar pruebas de funcionamiento, y corregir problemas.
\end{enumerate}
